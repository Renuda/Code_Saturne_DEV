%-------------------------------------------------------------------------------

% This file is part of Code_Saturne, a general-purpose CFD tool.
%
% Copyright (C) 1998-2021 EDF S.A.
%
% This program is free software; you can redistribute it and/or modify it under
% the terms of the GNU General Public License as published by the Free Software
% Foundation; either version 2 of the License, or (at your option) any later
% version.
%
% This program is distributed in the hope that it will be useful, but WITHOUT
% ANY WARRANTY; without even the implied warranty of MERCHANTABILITY or FITNESS
% FOR A PARTICULAR PURPOSE.  See the GNU General Public License for more
% details.
%
% You should have received a copy of the GNU General Public License along with
% this program; if not, write to the Free Software Foundation, Inc., 51 Franklin
% Street, Fifth Floor, Boston, MA 02110-1301, USA.

%-------------------------------------------------------------------------------
%===============================================
\section{Thermodynamics}
%===============================================


%===============================================
\subsection{Introduction}
%===============================================

The description of the thermodynamical of gaseous mixture is as close as
possible to the JANAF standard. The gases mixture is, often, considered as
composed of some {\textit{global} species (\emph{e.g.} oxidizer, products, fuel) each of
them being a mixture (with known ratio) of \textit{elementary} species
(oxygen, nitrogen, carbon dioxide, ...).

A tabulation of the enthalpy of both elementary and global species for some
temperatures is constructed (using JANAF polynoms) or read (if the user found
useful to define a global specie not simply related to elementary ones;
\emph{e.g.} unspecified hydrocarbon known by C, H, O, N, S analysis and heating
value). The thermodynamic properties of condensed phase are more simple:
formation enthalpy is computed using properties of gaseous products of
combustion with air (formation enthalpy of which is zero valued as O2 and N2
are reference state) and the lower heating value. The heat capacity of
condensed phase is assumed constant and it is a data the user has to enter (in
the corresponding data file dp\_FCP or dp\_FUE).
%%%%%%%%%%%%%%%%%%%%%%%%%%%%%%%%%%%%%%%%%%%%%%%%%%%%%%%%%%%%%%%%%%%%%%%%%%%%%%%%%%%%

%===============================================
\subsection{Gases enthalpy discretisation}
%===============================================

A table of gases (both elementary species and global ones) enthalpy for some
temperatures (the user choses number of points, temperature in dp\_*** file) is
computed (enthalpy of elementary species is computed using JANAF polynomia;
enthalpy for global species are computed by weighting of elementary ones) or red
(subroutine \fort{pptbht}). Then the entahlpy is supposed to be linear
vs. temperature in each temperature gap (i.e. continuous piece wise linear on
the whole temperature range). As a consequence, temperature is a linear function
of enthalpy; and a simple algorithm (subroutine \fort{cothht}) allows to
determine the enthalpy of a mixture of gases (for inlet conditions it is more
useful to indicate temperature and mass fractions) or to determine temperature
from enthalpy of the mixture and mass fractions of global species (common use in
every fluid particle, at every time step).

%===============================================
\subsection{Particles enthalpy discretisation}
%===============================================

Enthalpy of condensed material is rarely known. Usually, only the low heating
value and ultimate analysis are determined. So, using simple assumptions and the
enthalpy of known released species (after burning with air) the formation
enthalpy of coal or heavy oil can be computed. Assuming the thermal capacity is
constant for every condensed material a table can be built with ... two (more is
useless) temperatures, allowing the use of the same simple algorithm for
temperature-enthalpy conversion. When intermediate gaseous species (volatile or
vapour) are thermodynamically known, simple assumptions (\emph{e.g.}: char is
thermodynamically equivalent to pure carbon in reference state; ashes are
inert) allow one to deduce enthalpy for heterogeneous reactions (these energies have
not to be explicitely taken into account for the energy balance of particles).

